
% Default to the notebook output style

    


% Inherit from the specified cell style.




    
\documentclass[11pt]{article}

    
    
    \usepackage[T1]{fontenc}
    % Nicer default font (+ math font) than Computer Modern for most use cases
    \usepackage{mathpazo}

    % Basic figure setup, for now with no caption control since it's done
    % automatically by Pandoc (which extracts ![](path) syntax from Markdown).
    \usepackage{graphicx}
    % We will generate all images so they have a width \maxwidth. This means
    % that they will get their normal width if they fit onto the page, but
    % are scaled down if they would overflow the margins.
    \makeatletter
    \def\maxwidth{\ifdim\Gin@nat@width>\linewidth\linewidth
    \else\Gin@nat@width\fi}
    \makeatother
    \let\Oldincludegraphics\includegraphics
    % Set max figure width to be 80% of text width, for now hardcoded.
    \renewcommand{\includegraphics}[1]{\Oldincludegraphics[width=.8\maxwidth]{#1}}
    % Ensure that by default, figures have no caption (until we provide a
    % proper Figure object with a Caption API and a way to capture that
    % in the conversion process - todo).
    \usepackage{caption}
    \DeclareCaptionLabelFormat{nolabel}{}
    \captionsetup{labelformat=nolabel}

    \usepackage{adjustbox} % Used to constrain images to a maximum size 
    \usepackage{xcolor} % Allow colors to be defined
    \usepackage{enumerate} % Needed for markdown enumerations to work
    \usepackage{geometry} % Used to adjust the document margins
    \usepackage{amsmath} % Equations
    \usepackage{amssymb} % Equations
    \usepackage{textcomp} % defines textquotesingle
    % Hack from http://tex.stackexchange.com/a/47451/13684:
    \AtBeginDocument{%
        \def\PYZsq{\textquotesingle}% Upright quotes in Pygmentized code
    }
    \usepackage{upquote} % Upright quotes for verbatim code
    \usepackage{eurosym} % defines \euro
    \usepackage[mathletters]{ucs} % Extended unicode (utf-8) support
    \usepackage[utf8x]{inputenc} % Allow utf-8 characters in the tex document
    \usepackage{fancyvrb} % verbatim replacement that allows latex
    \usepackage{grffile} % extends the file name processing of package graphics 
                         % to support a larger range 
    % The hyperref package gives us a pdf with properly built
    % internal navigation ('pdf bookmarks' for the table of contents,
    % internal cross-reference links, web links for URLs, etc.)
    \usepackage{hyperref}
    \usepackage{longtable} % longtable support required by pandoc >1.10
    \usepackage{booktabs}  % table support for pandoc > 1.12.2
    \usepackage[inline]{enumitem} % IRkernel/repr support (it uses the enumerate* environment)
    \usepackage[normalem]{ulem} % ulem is needed to support strikethroughs (\sout)
                                % normalem makes italics be italics, not underlines
    

    
    
    % Colors for the hyperref package
    \definecolor{urlcolor}{rgb}{0,.145,.698}
    \definecolor{linkcolor}{rgb}{.71,0.21,0.01}
    \definecolor{citecolor}{rgb}{.12,.54,.11}

    % ANSI colors
    \definecolor{ansi-black}{HTML}{3E424D}
    \definecolor{ansi-black-intense}{HTML}{282C36}
    \definecolor{ansi-red}{HTML}{E75C58}
    \definecolor{ansi-red-intense}{HTML}{B22B31}
    \definecolor{ansi-green}{HTML}{00A250}
    \definecolor{ansi-green-intense}{HTML}{007427}
    \definecolor{ansi-yellow}{HTML}{DDB62B}
    \definecolor{ansi-yellow-intense}{HTML}{B27D12}
    \definecolor{ansi-blue}{HTML}{208FFB}
    \definecolor{ansi-blue-intense}{HTML}{0065CA}
    \definecolor{ansi-magenta}{HTML}{D160C4}
    \definecolor{ansi-magenta-intense}{HTML}{A03196}
    \definecolor{ansi-cyan}{HTML}{60C6C8}
    \definecolor{ansi-cyan-intense}{HTML}{258F8F}
    \definecolor{ansi-white}{HTML}{C5C1B4}
    \definecolor{ansi-white-intense}{HTML}{A1A6B2}

    % commands and environments needed by pandoc snippets
    % extracted from the output of `pandoc -s`
    \providecommand{\tightlist}{%
      \setlength{\itemsep}{0pt}\setlength{\parskip}{0pt}}
    \DefineVerbatimEnvironment{Highlighting}{Verbatim}{commandchars=\\\{\}}
    % Add ',fontsize=\small' for more characters per line
    \newenvironment{Shaded}{}{}
    \newcommand{\KeywordTok}[1]{\textcolor[rgb]{0.00,0.44,0.13}{\textbf{{#1}}}}
    \newcommand{\DataTypeTok}[1]{\textcolor[rgb]{0.56,0.13,0.00}{{#1}}}
    \newcommand{\DecValTok}[1]{\textcolor[rgb]{0.25,0.63,0.44}{{#1}}}
    \newcommand{\BaseNTok}[1]{\textcolor[rgb]{0.25,0.63,0.44}{{#1}}}
    \newcommand{\FloatTok}[1]{\textcolor[rgb]{0.25,0.63,0.44}{{#1}}}
    \newcommand{\CharTok}[1]{\textcolor[rgb]{0.25,0.44,0.63}{{#1}}}
    \newcommand{\StringTok}[1]{\textcolor[rgb]{0.25,0.44,0.63}{{#1}}}
    \newcommand{\CommentTok}[1]{\textcolor[rgb]{0.38,0.63,0.69}{\textit{{#1}}}}
    \newcommand{\OtherTok}[1]{\textcolor[rgb]{0.00,0.44,0.13}{{#1}}}
    \newcommand{\AlertTok}[1]{\textcolor[rgb]{1.00,0.00,0.00}{\textbf{{#1}}}}
    \newcommand{\FunctionTok}[1]{\textcolor[rgb]{0.02,0.16,0.49}{{#1}}}
    \newcommand{\RegionMarkerTok}[1]{{#1}}
    \newcommand{\ErrorTok}[1]{\textcolor[rgb]{1.00,0.00,0.00}{\textbf{{#1}}}}
    \newcommand{\NormalTok}[1]{{#1}}
    
    % Additional commands for more recent versions of Pandoc
    \newcommand{\ConstantTok}[1]{\textcolor[rgb]{0.53,0.00,0.00}{{#1}}}
    \newcommand{\SpecialCharTok}[1]{\textcolor[rgb]{0.25,0.44,0.63}{{#1}}}
    \newcommand{\VerbatimStringTok}[1]{\textcolor[rgb]{0.25,0.44,0.63}{{#1}}}
    \newcommand{\SpecialStringTok}[1]{\textcolor[rgb]{0.73,0.40,0.53}{{#1}}}
    \newcommand{\ImportTok}[1]{{#1}}
    \newcommand{\DocumentationTok}[1]{\textcolor[rgb]{0.73,0.13,0.13}{\textit{{#1}}}}
    \newcommand{\AnnotationTok}[1]{\textcolor[rgb]{0.38,0.63,0.69}{\textbf{\textit{{#1}}}}}
    \newcommand{\CommentVarTok}[1]{\textcolor[rgb]{0.38,0.63,0.69}{\textbf{\textit{{#1}}}}}
    \newcommand{\VariableTok}[1]{\textcolor[rgb]{0.10,0.09,0.49}{{#1}}}
    \newcommand{\ControlFlowTok}[1]{\textcolor[rgb]{0.00,0.44,0.13}{\textbf{{#1}}}}
    \newcommand{\OperatorTok}[1]{\textcolor[rgb]{0.40,0.40,0.40}{{#1}}}
    \newcommand{\BuiltInTok}[1]{{#1}}
    \newcommand{\ExtensionTok}[1]{{#1}}
    \newcommand{\PreprocessorTok}[1]{\textcolor[rgb]{0.74,0.48,0.00}{{#1}}}
    \newcommand{\AttributeTok}[1]{\textcolor[rgb]{0.49,0.56,0.16}{{#1}}}
    \newcommand{\InformationTok}[1]{\textcolor[rgb]{0.38,0.63,0.69}{\textbf{\textit{{#1}}}}}
    \newcommand{\WarningTok}[1]{\textcolor[rgb]{0.38,0.63,0.69}{\textbf{\textit{{#1}}}}}
    
    
    % Define a nice break command that doesn't care if a line doesn't already
    % exist.
    \def\br{\hspace*{\fill} \\* }
    % Math Jax compatability definitions
    \def\gt{>}
    \def\lt{<}
    % Document parameters
    \title{markdown-solutions}
    
    
    

    % Pygments definitions
    
\makeatletter
\def\PY@reset{\let\PY@it=\relax \let\PY@bf=\relax%
    \let\PY@ul=\relax \let\PY@tc=\relax%
    \let\PY@bc=\relax \let\PY@ff=\relax}
\def\PY@tok#1{\csname PY@tok@#1\endcsname}
\def\PY@toks#1+{\ifx\relax#1\empty\else%
    \PY@tok{#1}\expandafter\PY@toks\fi}
\def\PY@do#1{\PY@bc{\PY@tc{\PY@ul{%
    \PY@it{\PY@bf{\PY@ff{#1}}}}}}}
\def\PY#1#2{\PY@reset\PY@toks#1+\relax+\PY@do{#2}}

\expandafter\def\csname PY@tok@w\endcsname{\def\PY@tc##1{\textcolor[rgb]{0.73,0.73,0.73}{##1}}}
\expandafter\def\csname PY@tok@c\endcsname{\let\PY@it=\textit\def\PY@tc##1{\textcolor[rgb]{0.25,0.50,0.50}{##1}}}
\expandafter\def\csname PY@tok@cp\endcsname{\def\PY@tc##1{\textcolor[rgb]{0.74,0.48,0.00}{##1}}}
\expandafter\def\csname PY@tok@k\endcsname{\let\PY@bf=\textbf\def\PY@tc##1{\textcolor[rgb]{0.00,0.50,0.00}{##1}}}
\expandafter\def\csname PY@tok@kp\endcsname{\def\PY@tc##1{\textcolor[rgb]{0.00,0.50,0.00}{##1}}}
\expandafter\def\csname PY@tok@kt\endcsname{\def\PY@tc##1{\textcolor[rgb]{0.69,0.00,0.25}{##1}}}
\expandafter\def\csname PY@tok@o\endcsname{\def\PY@tc##1{\textcolor[rgb]{0.40,0.40,0.40}{##1}}}
\expandafter\def\csname PY@tok@ow\endcsname{\let\PY@bf=\textbf\def\PY@tc##1{\textcolor[rgb]{0.67,0.13,1.00}{##1}}}
\expandafter\def\csname PY@tok@nb\endcsname{\def\PY@tc##1{\textcolor[rgb]{0.00,0.50,0.00}{##1}}}
\expandafter\def\csname PY@tok@nf\endcsname{\def\PY@tc##1{\textcolor[rgb]{0.00,0.00,1.00}{##1}}}
\expandafter\def\csname PY@tok@nc\endcsname{\let\PY@bf=\textbf\def\PY@tc##1{\textcolor[rgb]{0.00,0.00,1.00}{##1}}}
\expandafter\def\csname PY@tok@nn\endcsname{\let\PY@bf=\textbf\def\PY@tc##1{\textcolor[rgb]{0.00,0.00,1.00}{##1}}}
\expandafter\def\csname PY@tok@ne\endcsname{\let\PY@bf=\textbf\def\PY@tc##1{\textcolor[rgb]{0.82,0.25,0.23}{##1}}}
\expandafter\def\csname PY@tok@nv\endcsname{\def\PY@tc##1{\textcolor[rgb]{0.10,0.09,0.49}{##1}}}
\expandafter\def\csname PY@tok@no\endcsname{\def\PY@tc##1{\textcolor[rgb]{0.53,0.00,0.00}{##1}}}
\expandafter\def\csname PY@tok@nl\endcsname{\def\PY@tc##1{\textcolor[rgb]{0.63,0.63,0.00}{##1}}}
\expandafter\def\csname PY@tok@ni\endcsname{\let\PY@bf=\textbf\def\PY@tc##1{\textcolor[rgb]{0.60,0.60,0.60}{##1}}}
\expandafter\def\csname PY@tok@na\endcsname{\def\PY@tc##1{\textcolor[rgb]{0.49,0.56,0.16}{##1}}}
\expandafter\def\csname PY@tok@nt\endcsname{\let\PY@bf=\textbf\def\PY@tc##1{\textcolor[rgb]{0.00,0.50,0.00}{##1}}}
\expandafter\def\csname PY@tok@nd\endcsname{\def\PY@tc##1{\textcolor[rgb]{0.67,0.13,1.00}{##1}}}
\expandafter\def\csname PY@tok@s\endcsname{\def\PY@tc##1{\textcolor[rgb]{0.73,0.13,0.13}{##1}}}
\expandafter\def\csname PY@tok@sd\endcsname{\let\PY@it=\textit\def\PY@tc##1{\textcolor[rgb]{0.73,0.13,0.13}{##1}}}
\expandafter\def\csname PY@tok@si\endcsname{\let\PY@bf=\textbf\def\PY@tc##1{\textcolor[rgb]{0.73,0.40,0.53}{##1}}}
\expandafter\def\csname PY@tok@se\endcsname{\let\PY@bf=\textbf\def\PY@tc##1{\textcolor[rgb]{0.73,0.40,0.13}{##1}}}
\expandafter\def\csname PY@tok@sr\endcsname{\def\PY@tc##1{\textcolor[rgb]{0.73,0.40,0.53}{##1}}}
\expandafter\def\csname PY@tok@ss\endcsname{\def\PY@tc##1{\textcolor[rgb]{0.10,0.09,0.49}{##1}}}
\expandafter\def\csname PY@tok@sx\endcsname{\def\PY@tc##1{\textcolor[rgb]{0.00,0.50,0.00}{##1}}}
\expandafter\def\csname PY@tok@m\endcsname{\def\PY@tc##1{\textcolor[rgb]{0.40,0.40,0.40}{##1}}}
\expandafter\def\csname PY@tok@gh\endcsname{\let\PY@bf=\textbf\def\PY@tc##1{\textcolor[rgb]{0.00,0.00,0.50}{##1}}}
\expandafter\def\csname PY@tok@gu\endcsname{\let\PY@bf=\textbf\def\PY@tc##1{\textcolor[rgb]{0.50,0.00,0.50}{##1}}}
\expandafter\def\csname PY@tok@gd\endcsname{\def\PY@tc##1{\textcolor[rgb]{0.63,0.00,0.00}{##1}}}
\expandafter\def\csname PY@tok@gi\endcsname{\def\PY@tc##1{\textcolor[rgb]{0.00,0.63,0.00}{##1}}}
\expandafter\def\csname PY@tok@gr\endcsname{\def\PY@tc##1{\textcolor[rgb]{1.00,0.00,0.00}{##1}}}
\expandafter\def\csname PY@tok@ge\endcsname{\let\PY@it=\textit}
\expandafter\def\csname PY@tok@gs\endcsname{\let\PY@bf=\textbf}
\expandafter\def\csname PY@tok@gp\endcsname{\let\PY@bf=\textbf\def\PY@tc##1{\textcolor[rgb]{0.00,0.00,0.50}{##1}}}
\expandafter\def\csname PY@tok@go\endcsname{\def\PY@tc##1{\textcolor[rgb]{0.53,0.53,0.53}{##1}}}
\expandafter\def\csname PY@tok@gt\endcsname{\def\PY@tc##1{\textcolor[rgb]{0.00,0.27,0.87}{##1}}}
\expandafter\def\csname PY@tok@err\endcsname{\def\PY@bc##1{\setlength{\fboxsep}{0pt}\fcolorbox[rgb]{1.00,0.00,0.00}{1,1,1}{\strut ##1}}}
\expandafter\def\csname PY@tok@kc\endcsname{\let\PY@bf=\textbf\def\PY@tc##1{\textcolor[rgb]{0.00,0.50,0.00}{##1}}}
\expandafter\def\csname PY@tok@kd\endcsname{\let\PY@bf=\textbf\def\PY@tc##1{\textcolor[rgb]{0.00,0.50,0.00}{##1}}}
\expandafter\def\csname PY@tok@kn\endcsname{\let\PY@bf=\textbf\def\PY@tc##1{\textcolor[rgb]{0.00,0.50,0.00}{##1}}}
\expandafter\def\csname PY@tok@kr\endcsname{\let\PY@bf=\textbf\def\PY@tc##1{\textcolor[rgb]{0.00,0.50,0.00}{##1}}}
\expandafter\def\csname PY@tok@bp\endcsname{\def\PY@tc##1{\textcolor[rgb]{0.00,0.50,0.00}{##1}}}
\expandafter\def\csname PY@tok@fm\endcsname{\def\PY@tc##1{\textcolor[rgb]{0.00,0.00,1.00}{##1}}}
\expandafter\def\csname PY@tok@vc\endcsname{\def\PY@tc##1{\textcolor[rgb]{0.10,0.09,0.49}{##1}}}
\expandafter\def\csname PY@tok@vg\endcsname{\def\PY@tc##1{\textcolor[rgb]{0.10,0.09,0.49}{##1}}}
\expandafter\def\csname PY@tok@vi\endcsname{\def\PY@tc##1{\textcolor[rgb]{0.10,0.09,0.49}{##1}}}
\expandafter\def\csname PY@tok@vm\endcsname{\def\PY@tc##1{\textcolor[rgb]{0.10,0.09,0.49}{##1}}}
\expandafter\def\csname PY@tok@sa\endcsname{\def\PY@tc##1{\textcolor[rgb]{0.73,0.13,0.13}{##1}}}
\expandafter\def\csname PY@tok@sb\endcsname{\def\PY@tc##1{\textcolor[rgb]{0.73,0.13,0.13}{##1}}}
\expandafter\def\csname PY@tok@sc\endcsname{\def\PY@tc##1{\textcolor[rgb]{0.73,0.13,0.13}{##1}}}
\expandafter\def\csname PY@tok@dl\endcsname{\def\PY@tc##1{\textcolor[rgb]{0.73,0.13,0.13}{##1}}}
\expandafter\def\csname PY@tok@s2\endcsname{\def\PY@tc##1{\textcolor[rgb]{0.73,0.13,0.13}{##1}}}
\expandafter\def\csname PY@tok@sh\endcsname{\def\PY@tc##1{\textcolor[rgb]{0.73,0.13,0.13}{##1}}}
\expandafter\def\csname PY@tok@s1\endcsname{\def\PY@tc##1{\textcolor[rgb]{0.73,0.13,0.13}{##1}}}
\expandafter\def\csname PY@tok@mb\endcsname{\def\PY@tc##1{\textcolor[rgb]{0.40,0.40,0.40}{##1}}}
\expandafter\def\csname PY@tok@mf\endcsname{\def\PY@tc##1{\textcolor[rgb]{0.40,0.40,0.40}{##1}}}
\expandafter\def\csname PY@tok@mh\endcsname{\def\PY@tc##1{\textcolor[rgb]{0.40,0.40,0.40}{##1}}}
\expandafter\def\csname PY@tok@mi\endcsname{\def\PY@tc##1{\textcolor[rgb]{0.40,0.40,0.40}{##1}}}
\expandafter\def\csname PY@tok@il\endcsname{\def\PY@tc##1{\textcolor[rgb]{0.40,0.40,0.40}{##1}}}
\expandafter\def\csname PY@tok@mo\endcsname{\def\PY@tc##1{\textcolor[rgb]{0.40,0.40,0.40}{##1}}}
\expandafter\def\csname PY@tok@ch\endcsname{\let\PY@it=\textit\def\PY@tc##1{\textcolor[rgb]{0.25,0.50,0.50}{##1}}}
\expandafter\def\csname PY@tok@cm\endcsname{\let\PY@it=\textit\def\PY@tc##1{\textcolor[rgb]{0.25,0.50,0.50}{##1}}}
\expandafter\def\csname PY@tok@cpf\endcsname{\let\PY@it=\textit\def\PY@tc##1{\textcolor[rgb]{0.25,0.50,0.50}{##1}}}
\expandafter\def\csname PY@tok@c1\endcsname{\let\PY@it=\textit\def\PY@tc##1{\textcolor[rgb]{0.25,0.50,0.50}{##1}}}
\expandafter\def\csname PY@tok@cs\endcsname{\let\PY@it=\textit\def\PY@tc##1{\textcolor[rgb]{0.25,0.50,0.50}{##1}}}

\def\PYZbs{\char`\\}
\def\PYZus{\char`\_}
\def\PYZob{\char`\{}
\def\PYZcb{\char`\}}
\def\PYZca{\char`\^}
\def\PYZam{\char`\&}
\def\PYZlt{\char`\<}
\def\PYZgt{\char`\>}
\def\PYZsh{\char`\#}
\def\PYZpc{\char`\%}
\def\PYZdl{\char`\$}
\def\PYZhy{\char`\-}
\def\PYZsq{\char`\'}
\def\PYZdq{\char`\"}
\def\PYZti{\char`\~}
% for compatibility with earlier versions
\def\PYZat{@}
\def\PYZlb{[}
\def\PYZrb{]}
\makeatother


    % Exact colors from NB
    \definecolor{incolor}{rgb}{0.0, 0.0, 0.5}
    \definecolor{outcolor}{rgb}{0.545, 0.0, 0.0}



    
    % Prevent overflowing lines due to hard-to-break entities
    \sloppy 
    % Setup hyperref package
    \hypersetup{
      breaklinks=true,  % so long urls are correctly broken across lines
      colorlinks=true,
      urlcolor=urlcolor,
      linkcolor=linkcolor,
      citecolor=citecolor,
      }
    % Slightly bigger margins than the latex defaults
    
    \geometry{verbose,tmargin=1in,bmargin=1in,lmargin=1in,rmargin=1in}
    
    

    \begin{document}
    
    
    \maketitle
    
    

    
    \hypertarget{ls-88---lab-2---markdown---solutions}{%
\section{L\&S 88 - Lab 2 - Markdown -
SOLUTIONS}\label{ls-88---lab-2---markdown---solutions}}

\emph{Lab adapted by Chris Pyles from
\href{https://github.com/jupyter/notebook/blob/master/docs/source/examples/Notebook/Working\%20With\%20Markdown\%20Cells.ipynb}{Working
With Markdown Cells}}

Markdown is an integral feature of Jupyter Notebooks; it is a formatting
language that allows you to create rich (that is, stylized) text.
Jupyter supports Github flavored Markdown, which has some nice features
like HTML output and syntax highlighting (we'll get to those later).
Markdown is based on HTML (HyperText Markup Language, the language that
webpages are written in) and is widely used across domains. It is an
integral part of Jupyter notebooks, and it is in Markdown that we write
the narrative that connects hard-to-follow code snippets.

In this lab, we'll go through an introduction to Markdown, beyond the
simple rich text that it will let you create, and cover some of its more
interesting characteristics. In particular, we'll cover things like code
blocks, tables, embedding HTML, and other cool features. You don't need
to know HTML for this lab, but you will be taught two tags (\texttt{div}
and \texttt{img}) as well as HTML id's.

Table of Contents:

\begin{enumerate}
\def\labelenumi{\arabic{enumi}.}
\tightlist
\item
  Section \ref{basis}
\item
  Section \ref{headings}
\item
  Section \ref{quotes}
\item
  Section \ref{code}
\item
  Section \ref{images}
\item
  Section \ref{conclusion}
\item
  Section \ref{submission}
\end{enumerate}

    \hypertarget{basis}{}

\hypertarget{the-basics}{%
\subsection{The Basics}\label{the-basics}}

Let's begin with some things you may remember from our short Markdown
intro from the last lab:

\begin{longtable}[]{@{}lll@{}}
\toprule
Syntax & Use & HTML Tag\tabularnewline
\midrule
\endhead
\texttt{\_} or \texttt{*} & Italicize &
\texttt{\textless{}em\textgreater{}}\tabularnewline
\texttt{**} & Bolden &
\texttt{\textless{}strong\textgreater{}}\tabularnewline
\texttt{\textasciigrave{}} & Code & N/A\tabularnewline
\texttt{{[}{]}()} & Hyperlink &
\texttt{\textless{}a\textgreater{}}\tabularnewline
\texttt{1.} & Ordered list & \texttt{\textless{}ol\textgreater{}},
\texttt{\textless{}li\textgreater{}}\tabularnewline
\texttt{*} & Unordered list & \texttt{\textless{}ul},
\texttt{\textless{}li\textgreater{}}\tabularnewline
\texttt{\#} & Headings & \texttt{\textless{}h1\textgreater{}},
\texttt{\textless{}h2\textgreater{}}, \ldots{}\tabularnewline
\texttt{-\/-\/-} & Line & N/A\tabularnewline
\bottomrule
\end{longtable}

To get literal characters such as \_ and *, append a
\texttt{\textbackslash{}} to the beginning of each:
\texttt{\textbackslash{}\_} or \texttt{\textbackslash{}*}. Using the
hyperlink syntax, you link either to external webpages or to local
files. For example, to link to the \href{./test.pdf}{test.pdf file} in
this directory, you would write
\texttt{{[}test.pdf\ file{]}(./test.pdf)}.

To create nested lists, you indent the line with the sublist element.
You can also mix ordered and unordered lists:

\begin{itemize}
\tightlist
\item
  List 1

  \begin{itemize}
  \tightlist
  \item
    Sublist 1

    \begin{itemize}
    \tightlist
    \item
      Subsublist 1
    \item
      Subsublist 2
    \end{itemize}
  \item
    Sublist 2
  \end{itemize}
\end{itemize}

\begin{enumerate}
\def\labelenumi{\arabic{enumi}.}
\tightlist
\item
  List 1

  \begin{enumerate}
  \def\labelenumii{\arabic{enumii}.}
  \tightlist
  \item
    Sublist 1

    \begin{itemize}
    \tightlist
    \item
      Subsublist 1
    \item
      Subsublist 2
    \end{itemize}
  \item
    Sublist 2
  \end{enumerate}
\end{enumerate}

\begin{Shaded}
\begin{Highlighting}[]
\NormalTok{* }\FloatTok{List 1}
\FloatTok{    * Sublist 1}
\FloatTok{        * Subsublist 1}
\FloatTok{        * Subsublist 2}
\FloatTok{    * Sublist 2}


\NormalTok{1. }\FloatTok{List 1}
\FloatTok{    1. Sublist 1}
\FloatTok{        * Subsublist 1}
\FloatTok{        * Subsublist 2}
\FloatTok{    2. Sublist 2}
\end{Highlighting}
\end{Shaded}

Note: if you make a list (either ordered or unordered), you need to
leave a blank line between the last item and the start of your next
paragraph, otherwise Markdown will render the next paragraph on the same
line or with the same indentation. For example:

\begin{Shaded}
\begin{Highlighting}[]
\NormalTok{1. }\FloatTok{first list item}
\FloatTok{2. second list item}
\FloatTok{(supposedly) new paragraph}
\end{Highlighting}
\end{Shaded}

\begin{enumerate}
\def\labelenumi{\arabic{enumi}.}
\tightlist
\item
  first list item
\item
  second list item (supposedly) new paragraph
\end{enumerate}

So, we can write \textbf{this} \emph{sentence} in
\href{https://google.com}{Markdown} using \texttt{this\ code}:

\begin{Shaded}
\begin{Highlighting}[]
\NormalTok{So, we can write **this** _sentence_ in }\OtherTok{[Markdown](https://google.com)}\NormalTok{ using }\BaseNTok{`this code`}\NormalTok{:}
\end{Highlighting}
\end{Shaded}

    \begin{center}\rule{0.5\linewidth}{\linethickness}\end{center}

\hypertarget{question-1}{%
\subsubsection{Question 1}\label{question-1}}

Write the Markdown code to recreate the text below in the following
Markdown cell. Do not edit the first and last two lines of that cell.
The URL for the hyperlink is \url{https://github.github.com/gfm/}.

\textless{}/center

    \begin{Shaded}
\begin{Highlighting}[]
\OtherTok{[_Github Flavored Markdown_](https://github.github.com/gfm/)}\NormalTok{ has some great features that we **intend** to make use of. Some of these features include:}
\NormalTok{* }\FloatTok{HTML integration}
\FloatTok{* syntax highlighting}
\FloatTok{* inline and block LaTeX}
\FloatTok{* beautiful and intuitive tables}

\NormalTok{We're going to cover these in the following order:}
\NormalTok{2. }\FloatTok{Headers & Block Quotes}
\FloatTok{    * Headers}
\FloatTok{    * }\BaseNTok{`div`}\FloatTok{ tags}
\FloatTok{3. Block Quotes & Comments}
\FloatTok{    * Block quotes}
\FloatTok{    * Comments}
\FloatTok{3. Code & LaTeX}
\FloatTok{    * inline code}
\FloatTok{    * code blocks}
\FloatTok{    * LaTeX}
\FloatTok{4. Images & Tables}
\FloatTok{    * Markdown images}
\FloatTok{    * HTML images}
\FloatTok{    * Tables}
\end{Highlighting}
\end{Shaded}

\begin{center}\rule{0.5\linewidth}{\linethickness}\end{center}

    \hypertarget{headings}{}

\hypertarget{headings-and-div-tags}{%
\subsection{\texorpdfstring{Headings and \texttt{div}
Tags}{Headings and div Tags}}\label{headings-and-div-tags}}

As you recall, we create headings with the \texttt{\#} character. Put
this character at the start of a line, with the number of characters
indicating the heading level (i.e.~a level 3 heading would begin with
\texttt{\#\#\#}). These correpsonding to the HTML \texttt{h1},
\texttt{h2}, etc. tags, and either will render in a Jupyter Notebook.

In order to create a table of contents such as the one in the first
cell, we use HTML tags which don't render to create subdivisions within
the notebook. This is usually done using HTML's \texttt{div} tag in
conjunction with HTML id's. To do this, we insert this HTML code into
the cell:

\begin{Shaded}
\begin{Highlighting}[]
\KeywordTok{<div}\OtherTok{ id=}\StringTok{"some-id"}\KeywordTok{></div>}
\end{Highlighting}
\end{Shaded}

(Make sure you leave a blank line after the HTML, otherwise headings
won't render!) It won't show up, but it does tell the browser that
you're using that the location of that \texttt{div} tag is where the
\texttt{some-id} section begins. To use these to make hyperlinks, we use
Markdown's hyperlink syntax:

\begin{Shaded}
\begin{Highlighting}[]
\OtherTok{[Some ID](#some-id)}
\end{Highlighting}
\end{Shaded}

If you look at the unrendered code for the table of contents at the top
of this notebook, you can see an example.

    \begin{center}\rule{0.5\linewidth}{\linethickness}\end{center}

\hypertarget{question-2}{%
\subsubsection{Question 2}\label{question-2}}

In the cell blow, write a level 5 heading. Put an anchor in the cell for
that heading and write a hyperlink to it. Make sure it doesn't conflict
with any of the anchors already in this notebook!

    \begin{Shaded}
\begin{Highlighting}[]
\NormalTok{<div id="l5heading"></div>}

\FunctionTok{##### This is my level 5 heading}

\OtherTok{[Level 5 Heading](#l5heading)}
\end{Highlighting}
\end{Shaded}

\begin{center}\rule{0.5\linewidth}{\linethickness}\end{center}

    \hypertarget{quotes}{}

\hypertarget{quotes-and-comments}{%
\subsection{Quotes and Comments}\label{quotes-and-comments}}

Block quotes are pretty simple in Markdown. To use block quotes,
Markdown provides the \texttt{\textgreater{}} operator. Begin the line
with \texttt{\textgreater{}} and a space, and get typing: \textgreater{}
This is a block quote.

\begin{Shaded}
\begin{Highlighting}[]
\NormalTok{>}\DataTypeTok{ This is a block quote.}
\end{Highlighting}
\end{Shaded}

Block quotes are quite helpful when you are writing up results and you
would like to borrow from someone who has already done work on what you
are studying. They are also a nice way to set text apart. (You may have
noticed that in the last lab, I set the main question apart from its
surrounding text using a block quote, even though I wasn't quoting
anyone.)

Just like any other coding language, Markdown allows you to leave
comments in your code. Just like in HTML, comments appear in the raw
version of the text, but do not render when the Markdown is formatted.
Markdown uses HTML syntax for comments:

\begin{Shaded}
\begin{Highlighting}[]
\NormalTok{<!--}\CommentTok{ comment here -->}
\end{Highlighting}
\end{Shaded}

You can have multiline comments using the same syntax:

\begin{Shaded}
\begin{Highlighting}[]
\NormalTok{<!--}\CommentTok{ this is a multi}
\CommentTok{line comment -->}
\end{Highlighting}
\end{Shaded}

Double click on this cell and look at the raw Markdown for an example of
Markdown comments. 

    \begin{center}\rule{0.5\linewidth}{\linethickness}\end{center}

\hypertarget{question-3}{%
\subsubsection{Question 3}\label{question-3}}

Recreate the \emph{The Question} section from last week's lab below. For
convenience, here is an image of it:

Notice the horizontal lines at the top and bottom. In Markdown, these
are made by putting three hyphens, \texttt{-\/-\/-}, on their own line.
Also recall that this is a section of the notebook, and as such there is
an invisible tag somewhere to allow you to link to it.

With regards to that fancy \(k\), it is made using inline LaTeX, which
is in the next section. For now, you can generate it using the syntax
\texttt{\$k\$}. Again, do not edit the first and last lines of the cell.

    \begin{Shaded}
\begin{Highlighting}[]
\NormalTok{---}
\NormalTok{<div id="question"></div>}

\FunctionTok{# The Question <!-- also allow level 2 heading) -->}
\NormalTok{The first part of developing a data-driven project is to decide what question you want to answer. The question needs to be specific, and it needs to be something you can develop a step-by-step approach for. With this notebook, I am going to use the }\BaseNTok{`movies`}\NormalTok{ Table to answer the following question:}

\NormalTok{>}\DataTypeTok{ Can we predict the genre of a movie based on its synopsis?}

\NormalTok{It will take a few steps to answer this question. The main methodology will be to create a test set and determine the frequency of different words in synopses within different genres, and then develop a $k$-nearest neighbors classifier based on this information. The over-arching workflow will look something like this:}
\NormalTok{1. }\FloatTok{Data preprocessing}
\FloatTok{2. Group movies by genre and look for recurring words in plots}
\FloatTok{3. Write a $k$-nearest neighbor classifier}
\FloatTok{4. Test the classifier and determine its accuracy}
\FloatTok{---}
\end{Highlighting}
\end{Shaded}

\begin{center}\rule{0.5\linewidth}{\linethickness}\end{center}

    \hypertarget{code}{}

\hypertarget{code-and-latex}{%
\subsection{Code and LaTeX}\label{code-and-latex}}

As you probably have noticed by now, we can write syntax in Markdown
that generates text which looks like code. There are two ways to do
this: inline code, and block code. Inline code appears within regular
text, and is generated by surrounding text with backticks,
\texttt{\textasciigrave{}}. It does \emph{not} support syntax
highlighting. In code blocks, the code is set aside from the text in a
new paragraph. This is generated by wrapping the code block with three
backticks. In Github flavored Markdown (which is what is in Jupyter
Notebooks), this also allows you to use syntax highlighting. (Ever
notice how in code cells some words are different colors or styles?
That's syntax highlighting.) To make use of this feature, write the name
of the language right after the backticks on the first line:

\texttt{}python def my\_func(x): return x

\begin{verbatim}

```python
def my_func(x):
    return x
```

```
```html
<div id="code"></div>
\end{verbatim}

\begin{Shaded}
\begin{Highlighting}[]
\KeywordTok{<div}\OtherTok{ id=}\StringTok{"code"}\KeywordTok{></div>}
\end{Highlighting}
\end{Shaded}

The languages supported include Python, Java, C, R, Julia, Ruby, bash,
HTML, CSS, and Markdown itself. See
\href{https://github.com/github/linguist/blob/master/lib/linguist/languages.yml}{this
YAML file} for a complete list of supported languages.

Markdown also supports inline and block LaTeX, which is a language that
renders mathematical expressions. For inline LaTeX, wrap the code in
\texttt{\$}. For block LaTeX, wrap in double dollar signs,
\texttt{\$\$}. (To use literal \textbackslash{}\$ characters in text,
append \texttt{\textbackslash{}\textbackslash{}} to the beginning of
each: \texttt{\textbackslash{}\textbackslash{}\$}.) We won't get into
the nitty gritty of LaTeX commands; if you need help, there are some
LaTeX code generators out there, such as
\href{https://www.codecogs.com/latex/eqneditor.php}{this one}. As an
example of LaTeX, here is a wonderful expression; it is the Ramanujan
approximation of \(\frac{1}{\pi}\):

\[\frac{1}{\pi} = \frac{2 \sqrt{2}}{9801} \sum \limits_{k=0}^\infty \frac{(4k)!(1193 + 26390k)}{(k!)^4 396^{4k}}\]

\begin{Shaded}
\begin{Highlighting}[]
\FunctionTok{\textbackslash{}frac}\NormalTok{\{1\}\{}\FunctionTok{\textbackslash{}pi}\NormalTok{\} = }\FunctionTok{\textbackslash{}frac}\NormalTok{\{2 }\FunctionTok{\textbackslash{}sqrt}\NormalTok{\{2\}\}\{9801\} }\FunctionTok{\textbackslash{}sum} \FunctionTok{\textbackslash{}limits}\NormalTok{_\{k=0\}^}\FunctionTok{\textbackslash{}infty} \FunctionTok{\textbackslash{}frac}\NormalTok{\{(4k)!(1193 + 26390k)\}\{(k!)^4 396^\{4k\}\}}
\end{Highlighting}
\end{Shaded}

    \begin{center}\rule{0.5\linewidth}{\linethickness}\end{center}

\hypertarget{question-4}{%
\subsubsection{Question 4}\label{question-4}}

Use LaTeX to recreate the following equation \emph{in its own block}.
Below that, use Markdown syntax highlighting to show the LaTeX code you
used. (In the code block portion, you can forego the dollar signs,
otherwise it won't render nicely.)

    \begin{Shaded}
\begin{Highlighting}[]

\NormalTok{$$f_X(x) = \textbackslash{}frac\{1\}\{\textbackslash{}sqrt\{2 \textbackslash{}pi \textbackslash{}sigma^2\} e^\{- \textbackslash{}frac\{1\}\{2\} \textbackslash{}left ( \textbackslash{}frac\{x - \textbackslash{}mu\}\{\textbackslash{}sigma\} \textbackslash{}right ) ^2 \}$$}

\BaseNTok{```latex}
\BaseNTok{f_X(x) = \textbackslash{}frac\{1\}\{\textbackslash{}sqrt\{2 \textbackslash{}pi \textbackslash{}sigma^2\} e^\{- \textbackslash{}frac\{1\}\{2\} \textbackslash{}left ( \textbackslash{}frac\{x - \textbackslash{}mu\}\{\textbackslash{}sigma\} \textbackslash{}right ) ^2 \}}
\BaseNTok{```}
\end{Highlighting}
\end{Shaded}

\begin{center}\rule{0.5\linewidth}{\linethickness}\end{center}

    \hypertarget{images}{}

\hypertarget{images-and-tables}{%
\subsection{Images and Tables}\label{images-and-tables}}

Markdown supports two different syntaxes for rendering images. The first
is native to Markdown, and is almost identical to the hyperlink format:

\begin{Shaded}
\begin{Highlighting}[]
\AlertTok{![alt text](path/to/image)}
\end{Highlighting}
\end{Shaded}

For images in the same directory as your notebook, you can just put in
\texttt{filename.png} or \texttt{./filename.png} (replacing \texttt{png}
with the file format). But if, for example, you had the following file
structure:

\begin{verbatim}
parent
| - notebook.ipynb
| assets
  | - picture.png
\end{verbatim}

your filepath would be \texttt{assets/picture.png}.

The other image syntax is the same as the HTML syntax, the \texttt{img}
tag. In HTML, \texttt{img} is a self-contained tag, meaning that it
doesn't require an ending \texttt{\textless{}/img\textgreater{}} tag.
This is the syntax:

\begin{Shaded}
\begin{Highlighting}[]
\KeywordTok{<img}\OtherTok{ src=}\StringTok{"path/to/image"} \KeywordTok{/>}
\end{Highlighting}
\end{Shaded}

In most cases, you can set the dimensions of an image in either case.
For the Markdown syntax, you add \texttt{={[}width{]}x{[}height{]}} to
the end of the filepath (note the space between them):

\begin{Shaded}
\begin{Highlighting}[]
\AlertTok{![alt text](path/to/image =200x100)}
\end{Highlighting}
\end{Shaded}

Unfortunately, Jupyter Notebooks will not render images if you resize
them using Markdown resizing syntax; in Jupyter, if you want an image to
be a specific size, you need to use HTML syntax. For HTML, you add the
\texttt{width} and \texttt{height} elements to the \texttt{img} tag:

\begin{Shaded}
\begin{Highlighting}[]
\KeywordTok{<img}\OtherTok{ src=}\StringTok{"path/to/img"}\OtherTok{ width=}\StringTok{"200px"}\OtherTok{ height=}\StringTok{"100px"} \KeywordTok{/>}
\end{Highlighting}
\end{Shaded}

For examples of these image tags in action, take a look at the raw code
for the question cells.

The last part of Markdown you need to know is how to make a table. While
Markdown supports HTML tables, we'll use Markdown's native table syntax
because the HTML syntax is long and doesn't render nicely. Here is how
it looks:

\begin{Shaded}
\begin{Highlighting}[]
\NormalTok{| Column 1 | Column 2 | Column 3| etc. |}
\NormalTok{|-----|-----|-----|-----|}
\NormalTok{| Row 1 | entry 1 | entry 2 | etc. |}
\NormalTok{| Row 2 | entry 1 | entry 2 | etc. |}
\NormalTok{| etc. | etc. | etc. | etc. |}
\end{Highlighting}
\end{Shaded}

\begin{longtable}[]{@{}llll@{}}
\toprule
Column 1 & Column 2 & Column 3 & etc.\tabularnewline
\midrule
\endhead
Row 1 & entry 1 & entry 2 & etc.\tabularnewline
Row 2 & entry 1 & entry 2 & etc.\tabularnewline
etc. & etc. & etc. & etc.\tabularnewline
\bottomrule
\end{longtable}

Make sure that you don't forget the row of dashes between pipes,
otherwise Markdown won't recognize it as a table.

    \begin{center}\rule{0.5\linewidth}{\linethickness}\end{center}

\hypertarget{question-4}{%
\subsubsection{Question 4}\label{question-4}}

Write the code to render the image \texttt{compsci\_cat.jpg} with width
600px. (\emph{Hint:} Jupyter Notebooks will automatically rescale the
height of an image if you only specify the width.)

Below your image code, create a table of ice cream flavors (Vanilla,
Chocoloate, Strawberry, Neopolitan) and your rating of them on a scale
of 1 (disgusting) to 5 (amazing).

    \begin{Shaded}
\begin{Highlighting}[]
\NormalTok{<img src="compsci_cat.jpg" width="600px" />}

\NormalTok{| Flavor | Rating |}
\NormalTok{|-----|-----|}
\NormalTok{| Vanilla | 5 |}
\NormalTok{| Chocolate | 5 |}
\NormalTok{| Strawberry | 5 |}
\NormalTok{| Neopolitan | 5 |}
\end{Highlighting}
\end{Shaded}

\begin{center}\rule{0.5\linewidth}{\linethickness}\end{center}

    \hypertarget{conclusion}{}

\hypertarget{conclusions}{%
\subsection{Conclusions}\label{conclusions}}

This notebook is a pretty comprehensive introduction to Markdown. While
you are by no means finished learning it, these skills are sufficient to
allow you to do well with Jupyter Notebooks and Github. 

    \hypertarget{submission}{}

\hypertarget{submission}{%
\subsection{Submission}\label{submission}}

To submit this lab, you will use Github. Start by initliazing a new
repository. Use Markdown to write a README (nothing too fancy) that
explains what is in the repo. Finally, upload this lab and \emph{all the
other files you have in this folder on DataHub} to the repo. If you do
not upload all the other files, your notebook will not render correctly,
because the supporting files will be missing. Here is a list of the
files you need to upload: * \texttt{markdown-intro.ipynb} *
\texttt{compsci\_cat.jpg} * \texttt{q1.png} * \texttt{q2.png} *
\texttt{q3.png} * \texttt{test.pdf}


    % Add a bibliography block to the postdoc
    
    
    
    \end{document}
